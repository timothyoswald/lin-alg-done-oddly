\chapter{Inner Products}

\oldsection{Inner Products}
\begin{definition}
    Let $V$ be a vector space over $\mathbb{R}$. A valid inner product of $V$ satisfies the following four properties:
\begin{enumerate}
    \item $\langle\vec{u},\vec{v}\rangle=\langle\vec{v},\vec{u}\rangle$ for all $\vec{v},\vec{u}\in V$ (Symmetry)
    \item $\langle a\vec{x}+b\vec{y},\vec{z}\rangle=a\langle\vec{x},\vec{z}\rangle+b\langle\vec{y},\vec{z}\rangle$ for all $a,b\in\mathbb{R}$ and $\vec{x},\vec{y},\vec{z}\in V$ (Linearity)
    \item $\langle\vec{v},\vec{v}\rangle\geq 0$ for all $\vec{v}\in V$ (Non-negativity)
    \item $\langle\vec{v},\vec{v}\rangle=0$ if and only if $\vec{v}=\vec{0}$ (Non-degeneracy)
\end{enumerate}
\end{definition}
\begin{definition}
    An inner product space is a vector space $V$ along with an inner product that satisfies the aforementioned conditions.
\end{definition}
\begin{example}
    Consider $\mathbb{R}^n$. Let $\vec{x}=(x_1, x_2,\ldots, x_n)$ and $\vec{y}=(y_1,y_2,\ldots,y_n)$. One example of an inner product is the dot product. This is often notated and defined as $\langle\vec{x},\vec{y}\rangle=\vec{x}\cdot\vec{y}=\sum_{i=1}^n x_iy_i$.

    Let's check the four conditions of inner products. Symmetry and Linearity follow from the fact that addition and multiplication are commutative. Let $\vec{v}\in V$. We have that $\vec{v}\cdot\vec{v}=\sum_{i=1}^n v_iv_i=\sum_{i=1}^n v_i^2$. This satisfies, non-negativity since squares are always non-negative. If $\vec{v}=\vec{0}$, each $v_i=0$ so the whole summation equals $0$. Assume $\langle\vec{v},\vec{v}\rangle=\vec{v}\cdot\vec{v}=\sum_{i=1}^n v_i^2=0$. Since $v_i^2\geq 0$, we must have that $v_i=0$ for all $i$ since otherwise we would have a positive sum. So we have non-degeneracy as well.
\end{example}
\begin{definition}
    Let $V$ be an inner product space and $\vec{v}\in V$. We define the norm of $\vec{v}$ to be $||\vec{v}||=\sqrt{\langle\vec{v},\vec{v}\rangle}$.
\end{definition}
\begin{example}
    Consider $\mathbb{R}^n$ with the dot product. Let $\vec{v}\in\mathbb{R}^n$. We have that $||\vec{v}||=\sqrt{\vec{v}\cdot\vec{v}}=\sqrt{\sum_{i=1}^n v_i^2}=\sqrt{v_1^2+v_2^2+\ldots+v_n^2}$.
\end{example}
\begin{theorem}
    Let $V$ be an inner product space and $\vec{v},\vec{w}\in V$. The following inequality always holds:
    $$|\langle\vec{v},\vec{w}\rangle|\leq||\vec{v}||||\vec{w}||$$
    This is known as the Cauchy-Schwartz Inequality.
\end{theorem}
\begin{theorem}
    Let $V$ be an inner product space and $\vec{v},\vec{w}\in V$. The following inequality always holds:
    $$||\vec{v}+\vec{w}||\leq||\vec{v}||+||\vec{w}||$$
    This is known as the Triangle Inequality.
\end{theorem}
\begin{proof}
    \begin{align*}
        ||\vec{v}+\vec{w}||^2&=(\sqrt{\langle\vec{v}+\vec{w},\vec{v}+\vec{w}\rangle})^2\\
        &=\langle\vec{v}+\vec{w},\vec{v}+\vec{w}\rangle\\
        &=\langle\vec{v},\vec{v}\rangle+\langle\vec{v},\vec{w}\rangle+\langle\vec{w},\vec{v}\rangle+\langle\vec{w},\vec{w}\rangle\tag{by Linearity of Inner Product}\\
        &=||\vec{v}||^2+\langle\vec{v},\vec{w}\rangle+\langle\vec{w},\vec{v}\rangle+||\vec{w}||^2\\
        &=||\vec{v}||^2+\langle\vec{v},\vec{w}\rangle+\langle\vec{v},\vec{w}\rangle+||\vec{w}||^2\tag{by Symmetry of Inner Product}\\
        &\leq||\vec{v}||^2+|\langle\vec{v},\vec{w}\rangle+\langle\vec{v},\vec{w}\rangle|+||\vec{w}||^2\\
        &\leq||\vec{v}||^2+|\langle\vec{v},\vec{w}\rangle|+|\langle\vec{v},\vec{w}\rangle|+||\vec{w}||^2\\
        &\leq||\vec{v}||^2+||\vec{v}||||\vec{w}||+||\vec{v}||||\vec{w}||+||\vec{w}||^2\tag{by Cauchy Schwartz Inequality}\\
        &\leq||\vec{v}||^2+2||\vec{v}||||\vec{w}||+||\vec{w}||^2
    \end{align*}
    Take the square root of both sides to get the Triangle Inequality.
\end{proof}
\begin{definition}
    Let $V$ be an inner product space and $\vec{v},\vec{w}\in V$. We say that $\vec{v}$ and $\vec{w}$ are orthogonal, denoted $\vec{v}\perp\vec{w}$, if $\langle\vec{v},\vec{w}\rangle=0$.
\end{definition}
\begin{definition}
    Let $\mathcal{L}=(\vec{v}_1,\vec{v}_2,\ldots,\vec{v}_n)$ be a list of vectors. We say that $\mathcal{L}$ is an orthogonal list if $\vec{v}_i\perp\vec{v}_j$ for all $i,j\in [n]$.
\end{definition}
\begin{theorem}
    Let $\mathcal{L}$ be an orthogonal list such that $\vec{0}\notin \mathcal{L}$. Then $\mathcal{L}$ is linearly independent.
\end{theorem}
\begin{definition}
    Let $\mathcal{L}=(\vec{v}_1,\vec{v}_2,\ldots,\vec{v}_n)$ be a list of vectors. We say that $\mathcal{L}$ is an orthonormal list if $\mathcal{L}$ is an orthogonal list and $||\vec{v}_i||=1$ for $1\leq i\leq n$.
\end{definition}
\begin{definition}
    Let $\mathcal{L}=(\vec{v}_1,\ldots,\vec{v}_k)$ be an orthonormal list. We define the projection of a $\vec{u}$ onto $\mathcal{L}$ as $proj_\mathcal{L}(\vec{u})=\sum_{i=1}^k\langle\vec{u},\vec{v}_i\rangle\vec{v}_i$
\end{definition}
\begin{theorem}
    Let $\mathcal{L}=(\vec{v}_1,\ldots,\vec{v}_k)$ be an orthonormal list. Then $\vec{x}-proj_\mathcal{L}(\vec{x})\perp \vec{v}_i$ for all $1\leq i\leq k$.
\end{theorem}
\begin{proof}
    \begin{align*}
        \langle \vec{x}-proj_\mathcal{L}(\vec{x}), \vec{v}_i\rangle&=\langle\vec{x},\vec{v}_i\rangle-\langle proj_\mathcal{L}(\vec{x}),\vec{v}_i\rangle\tag{by Linearity}\\
        &=\langle\vec{x},\vec{v}_i\rangle-\langle\sum_{j=1}^k\langle\vec{x},\vec{v}_j\rangle\vec{v}_j,\vec{v}_i\rangle\\
        &=\langle\vec{x},\vec{v}_i\rangle-\sum_{j=1}^k\langle\vec{x},\vec{v}_j\rangle\langle\vec{v}_j,\vec{v}_i\rangle\tag{by Linearity}
    \end{align*}
    First, consider the case when $j=i$. We observe that $\langle\vec{v}_i,\vec{v}_i\rangle=||\vec{v}_i||^2$ and since $\mathcal{L}$ is orthonormal we know that $||\vec{v}_i||=1$. Thus, $\langle\vec{v}_i,\vec{v}_i\rangle=1$. When $i\neq j$, then $\langle\vec{v}_j,\vec{v}_i\rangle=0$ since $\mathcal{L}$ is orthogonal.
    \begin{align*}
        \langle\vec{x},\vec{v}_i\rangle-\sum_{j=1}^k\langle\vec{x},\vec{v}_j\rangle\langle\vec{v}_j,\vec{v}_i\rangle=\langle\vec{x},\vec{v}_i\rangle-\langle\vec{x},\vec{v}_i\rangle=0
    \end{align*}
\end{proof}
\begin{definition}
    The Gram-Schmidt Process is a way of finding an orthonormal list $\mathcal{O}$ given a list of vectors $\mathcal{L}$ such that $span(\mathcal{O})=span(\mathcal{L})$.

    Let $\mathcal{O}_i$ denote the list $\mathcal{O}$ at step $i$. Let $\mathcal{L}_i$ denote the list of the first $i$ vectors of $\mathcal{L}$. We initialize $\mathcal{O}_0=\emptyset$. Note that initially, $span(\mathcal{O}_0)=span(\mathcal{L}_0)$ and $\mathcal{O}_{i-1}$ is orthonormal.

    At step $i$, look at the $i$th vector in $\mathcal{L}$ denoted $\vec{v}_i$. We assume that up to this point, we have constructed $\mathcal{O}_{i-1}$ such that $span(\mathcal{O}_{i-1})=span(\mathcal{L}_{i-1})$.

    Case 1: If $\vec{v}_i\in span(\mathcal{O}_{i-1})$, do nothing and move on to the $i+1$th vector.

    Case 2: If $\vec{v}_i\notin span(\mathcal{O}_{i-1})$, then we want to add it to make $\mathcal{O}_{i-1}$ into an orthonormal list with a larger span. To ensure the new vector is orthogonal to $\mathcal{O}_{i-1}$, we will use the projection. We will add $\frac{\vec{v}_i-proj_{\mathcal{O}_{i-1}}(\vec{v}_i)}{||\vec{v}_i-proj_{\mathcal{O}_{i-1}}(\vec{v}_i)||}$. The numerator is simply a vector that is orthogonal to $\mathcal{O}_{i-1}$ as we proved with projections. We divide by its norm to make this vector into a unit vector.
\end{definition}
\section{Exercises 7}
\begin{exercise}
    Let $\vec{x},\vec{y}\in\R^3$. Prove or disprove that $\langle\vec{x},\vec{y}\rangle=2x_1y_1+3x_2y_2+5x_3y_3$ where $\vec{x}=(x_1, x_2,x_3)$ and $\vec{y}=(y_1,y_2,y_3)$ is an inner product
\end{exercise}
\begin{exercise}
    Prove that the Cauchy-Schwartz Inequality has equality if and only if one of the vectors is a scalar of the other.
\end{exercise}
\begin{exercise}
    Perform the Gram-Schmidt Process on the following vectors:
    $$\vec{v}_1=\begin{pmatrix}
        1 \\ 1
    \end{pmatrix}\quad\quad\vec{v}_2=\begin{pmatrix}
        1 \\ -1
    \end{pmatrix}$$
\end{exercise}
\begin{exercise}
    Perform the Gram-Schmidt Process on the following vectors:
    $$\vec{v}_1=\begin{pmatrix}
        1 \\ 1 \\ 0
    \end{pmatrix}\quad\quad\vec{v}_2=\begin{pmatrix}
        1 \\ 0 \\ 1
    \end{pmatrix}\quad\quad\vec{v}_3=\begin{pmatrix}
        0 \\ 1 \\ 1
    \end{pmatrix}$$
\end{exercise}