\chapter{Preface}

From what I know, there are two "correct" ways to teach linear algebra: \textit{Linear Algebra Done Right} and \textit{Linear Algebra Done Wrong}. These are named after the books written about them by Sheldon Axler and Sergei Treil respectively. The difference between the two books is the use of matrices. In \textit{Linear Algebra Done Right}, a focus on linear operators/transformations is presented to justify all the elements of linear algebra one may be familiar with, with little attention to matrices. \textit{Linear Algebra Done Wrong} wants a more practical understanding of linear algebra and introduces matrices early in order to justify many other elements of linear algebra. Both are completely valid ways of teaching linear algebra and they simply serve different audiences. These notes will largely be based on \textit{Linear Algebra Done Wrong} with a few elements of \textit{Linear Algebra Done Right} here and there where intuition for a topic is deemed necessary. So we will only discuss finite dimensional vector spaces and will use matrices to justify many of the ideas such as determinants and eigenvalues.